\mbox{{\tiny \textcolor{white}{.,.,.}\hspace{-5.0ex}}}\mbox{{\tiny \textcolor{white}{.,.,.}\hspace{-5.0ex}}}% Created 2012-01-10 Tue 05:36
\documentclass[11pt]{article}
\usepackage[utf8]{inputenc}
\usepackage[T1]{fontenc}
\usepackage{fixltx2e}
\usepackage{graphicx}
\usepackage{longtable}
\usepackage{float}
\usepackage{wrapfig}
\usepackage{soul}
\usepackage{textcomp}
\usepackage{marvosym}
\usepackage{wasysym}
\usepackage{latexsym}
\usepackage{amssymb}
\usepackage{hyperref}
\tolerance=1000
\usepackage{xcolor}
\usepackage{listings}
\usepackage{setspace}
\usepackage{nomencl}
\makenomenclature
\providecommand{\alert}[1]{\textbf{#1}}

\title{Chapter 2: Background}
\author{Xin Yang}
\date{2012-01-10}
\hypersetup{
  pdfkeywords={},
  pdfsubject={},
  pdfcreator={Emacs Org-mode version 7.8.02}}

\begin{document}

\maketitle

\setcounter{tocdepth}{3}
\tableofcontents
\vspace*{1cm}

\printnomenclature
\setstretch{1.5}

\section{Introduction}
\label{sec-1}
\label{sec:ch2-introduction}

This chapter is continued with an introduction of the human auditory
system and some widely-accepted auditory models. Instead of providing
a lot of detailed information about the physiology and psychology,
this chapter emphasises on the behavioural characteristics of the
components in the auditory system in order to discover some
interesting functionalities of these components in terms of signal
processing. The objectives of this chapter are to:

\begin{enumerate}
\item Get a general understanding of the human auditory system and its
   power in the acoustic signal processing.
\item Assess some widely-accepted models of physiological components that
   comprise the auditory periphery in terms of their behavioural
   accuracy and their correspondence to known anatomical structure.
\item Select appropriate models which may be possible to be implemented
   as a real-time neuromorphic auditory periphery. These models will
   be investigated in later chapters.
\end{enumerate}
\section{Auditory System}
\label{sec-2}
\label{sec:ch2-auditory-system}

In this section, the anatomical structure of the human auditory system
is introduced. These physiological and psychophysical facts are
references in the real world for the modelling and implementation in
later chapters.

For different roles in the acoustic signal processing, the auditory
system comprises two parts \cite{Pickles2008}: the auditory periphery,
which pre-processes the incoming sound waves to generate neural
spikes, and the auditory brainstem, which begins neural processing and
finally leads to auditory sensations. In the past decades, significant
details about individual components of the auditory periphery are
revealed; while a clear understanding of the operation and purpose of
the cells within the auditory brainstem is still not available.
Therefore, this thesis emphasises on the auditory periphery.

It is convenient to consider the auditory periphery as consisting of
three divisions, the outer ear, the middle ear and the inner ear \footnote{Ext: ``It's yours? Offer a reference!'' }.
Figure \ref{fig:ear-periphery} is an illustration of the general
structure of the human auditory periphery.

\begin{figure}[htb]
\centering
\includegraphics[width=\textwidth]{./Chapter2Figs/ear-periphery.png}
\caption{\label{fig:ear-periphery}The three divisions of the human auditory periphery \cite{BritannicaEarPeriphery}.}
\end{figure}
\subsection{Outer Ear}
\label{sec-2-1}
\label{sec:ch2-outer-ear}

The externally visible portion of the outer ear is the auricle (or
pinna), and the portion ``inside the head'' is the external auditory
canal (or ear canal). The outer ear collects the mechanical sound
energy, and transmits the sound wave oscillations to the tympanic
membrane (or eardum) of the middle ear.
\subsection{Middle Ear}
\label{sec-2-2}
\label{sec:ch2-middle-ear}

The middle ear consists of three very small bones (or ossicles): the
malleus (or hammer), the incus (or anvil) and the stapes (or stirrup).
The malleus is attached to the other side of the eardrum and its
vibration is transmitted through the incus to the stapes. The stapes
motion impinges on the oval window, and cause the movements of
the fluids held within the inner ear. The middle ear translates air
vibrations into pressure waves in the fluids, and matches the
different impedances between the air and fluid without a great energy
loss which would otherwise occur \cite{Zwicker1999}.
\subsection{Inner Ear}
\label{sec-2-3}
\label{sec:ch2-inner-ear}

The snail-shaped inner ear (or cochlea), as shown in
Figure \ref{fig:cochlea}, is embedded in the extremely hard temporal
bone. There are three chambers/ducts inside the cochlea, called
scalae, which are filled with fluids. The three scalae are divided by
the Reissner's membrane and the basilar membrane (BM). The organ of
Corti (OC) is located on the BM, and contains two types of hair-cells:
inner hair cells (IHCs) and outer hair cells (OHCs).

\nomenclature{BM}{Basilar membrane}
\nomenclature{OC}{Organ of Corti}
\nomenclature{IHC}{Inner hair cell}
\nomenclature{OHC}{Outer hair cell}
\nomenclature{AN}{Auditory nerve}

% \begin{figure}[htb]
% \centering
% \includegraphics[width=\textwidth]{./Chapter2Figs/cochlea.png}
% \caption{\label{fig:cochlea}The cochlea and the ducts inside \cite{BritannicaEarBones}.}
% \end{figure}
\subsubsection{Basilar Membrane}
\label{sec-2-3-1}
\label{sec:ch2-basilar-membrane}

The cochlear fluid movements are conveyed to the BM within the
cochlea. Along the BM, where the stapes impinges on the oval window is
the base, and the far end is the apex. Near the base, the
BM is relatively narrow and stiff; while near the apex it is wider and
less stiff. BM acts like a frequency analyser, with the base part
responding to high frequencies, and the apical part responding to
lower frequencies, as shown in Figure \ref{fig:bm}. This
characteristic of BM is known as frequency selectivity.

\begin{figure}[htb]
\centering
\includegraphics[width=\textwidth]{./Chapter2Figs/bm.png}
\caption{\label{fig:bm}The frequency analysis of sound by the basilar membrane \cite{BritannicaBMfreq}.}
\end{figure}

BM also has many nonlinear characteristics, which are very
interesting and meaningful for the hearing system, such as:
\begin{itemize}

\item item Compression\\
\label{sec-2-3-1-1}%
The vibration magnitude of a BM site grows compressively with
increasing sound levels. This nonlinear characteristic does not only
protects the BM from the impact of very loud sound, but also helps
to increase the hearing resolution in an quite environment.


\item Level-Dependence\\
\label{sec-2-3-1-2}%
The characteristic frequency (CF) of basal BM sites, which respond to
high frequencies, decrease when the sound level increases, while the
bandwidths of all BM sites increase.


\item Suppression\\
\label{sec-2-3-1-3}%
When a second tone with a different frequency presents after the
first tone, the response to the first tone decreases. This is a
great help for the listener to realise the new tone.


\item Distortion\\
\label{sec-2-3-1-4}%
When BM response to a pair (or more) of tones with different
frequencies, its response contains a distortion product with the
frequency between the original two tones. For example, a third tone
(800Hz) would appear, when the original two tones (1000Hz and
1200Hz) play. \footnote{Ext: ``How? clarify this.'' }

These nonlinear characteristics have received a lot attention from
physiologists and psychophysists who developed various models of
BM to simulate these phenomena. An important aspect of this research
also lays in here, since these characteristics are quite useful. If a
real-time neuromorphic auditory periphery could represent these
behaviour, lots of applications  would emerge, such as a real-time
active speech enhancement device which is based on the compression and
suppression characteristics of the real BM.

\end{itemize} % ends low level
\subsubsection{Hair Cells}
\label{sec-2-3-2}
\label{sec:ch2-hair-cells}

The hair cells are arranged as one row of IHCs on the inner side of
the OC, and three parallel rows of OHCs near the middle, as shown in
Figure \ref{fig:corti}. Each hair cell has a lot of stereocilia or
hairs. The most important function of IHCs is to transform mechanical
movements to electrochemistric process which stimulates the AN fibres
by releasing neurotransmitters. This transformation happens when the
stereocilia of an IHC bend sufficiently in a specific direction (not
the other). This asymmetric gating led to the well-known description
of the IHC as a half-wave rectifier (HWR).

\nomenclature{HWR}{Half-wave rectifier}

\begin{figure}[htb]
\centering
\includegraphics[width=0.9\textwidth]{./Chapter2Figs/corti.png}
\caption{\label{fig:corti}Structure of the organ of Corti on the basilar membrane\cite{BritannicaOrganOfCorti}.}
\end{figure}

The main properties of the inner hair cell, as discovered through
physiological studies are:

\begin{itemize}
\item Onset and steady-state response to stimuli,
\item Two-component adaptation,
\item Recovery of spontaneous activity after stimulus offset.
\end{itemize}

\begin{figure}[htb]
\centering
\includegraphics[width=\textwidth]{./Chapter2Figs/ihc-response-trans.eps}
\caption{\label{fig:ihc-response}The amount of released neurotransmitters by an IHC to tones with different intensities.}
\end{figure}

These characteristics are best observed by looking at the response of
a real IHC to the application of 2 tones of different intensities, as
illustrated in Figure \ref{fig:ihc-response}. In the figure, the spike
rate produced in the auditory nerve by the IHC is seen, in the case of
applying a tone of intensities 30dB and 40dB. The greater the amount
of transmitter in the cleft the greater the spiking activity is caused
in the auditory nerve fibres. The permeability of the cell membrane is
related to the intensity of hair cell's input. The tones are applied
in regions B and C but no stimulus is applied in regions A and D.
Regions A of the responses represent the spontaneous activity of the
cell when no stimulus is applied. In region B the onset response of
the cell and the two-component adaptation can be seen. The magnitude
of the onset response is related to the intensity of the stimulus.
This onset is caused by a sudden increase in the membrane permeability
combined with full reservoirs. The two-component adaptation consists
of a rapid component, causing the steep decline from the onset
response, and a slower component, which contributes to the gentle
final stage of adaptation to the steady state response. Regions C show
that the steady state response is similar for both stimulus
intensities. This demonstrates the automatic gain control (AGC)
mechanism inherent within the hair cell. This automatic gain control
is caused by a balance between the depleted reservoirs (tending to
decrease the level of activity) and the increase in the membrane
permeability (tending to increase the level of activity). Regions D
show the recovery of the hair cell back to its spontaneous level of
activity. It can be seen that after the application of a high
intensity stimulus the hair cell takes longer to recover. This is
caused by heavily depleted reservoirs taking time to refill.

\nomenclature{AGC}{Automatic gain control}

There is another characteristic of the hair cell, phase locking, which
occurs when low frequency tones are applied (lower than 2kHz in
humans). This effect results in the hair cell stimulating tone. The
overall effect is that the temporal information in the signal is
preserved within the auditory nerve.
\subsubsection{Auditory Nerve}
\label{sec-2-3-3}
\label{sec:ch2-auditory-nerve}

The auditory nerve (AN) is made up of a collection of nerve fibres. When
neurotransmitters are released into the synaptic cleft between an IHC
and an AN fibre, a spike may be generated in an AN fibre. These
spikes along the AN fibres are transmitted to various cells within the
auditory brainstem. For human, there are about 30,000 fibres
associated with each cochlea. The AN fibres are often divided into
three categories according to their spontaneous firing rates:

\begin{enumerate}
\item High-spontaneous rate (HSR) fibres (60\% of all nerve fibers,
   firing rate > 18 spikes/second)
\item Medium-spontaneous rate (MSR) fibres (25\% of all nerve
   fibres, firing rate are 0.5--18 spikes/second)
\item Low-spontaneous rate (LSR) fibres (15\% of all nerve fibers,
   firing rate < 0.5 spikes/second)
\end{enumerate}

\nomenclature{HSR}{High-spontaneous rate}
\nomenclature{MSR}{Medium-spontaneous rate}
\nomenclature{LSR}{Low-spontaneous rate}
\subsection{Auditory Brainstem}
\label{sec-2-4}
\label{sec:ch2-auditory-brainstem}
\subsubsection{Neurons}
\label{sec-2-4-1}
\label{sec:ch2-neurons}

There are various types of neurons within the auditory brainstem,
which has the ability to transmit information by electrical and
chemical signalling. A typical neuron is illustrated in
Figure \ref{fig:neuron}. The input connectors are the dendrites; the
output connector is the axon, and the cell body is the soma. The
synapse is a junction between two neurons where the transmission of
electrical and/or chemical signals happens.

\begin{figure}[htb]
\centering
\includegraphics[width=0.8\textwidth]{./Chapter2Figs/neuron.eps}
\caption{\label{fig:neuron}Structure of a typical neuron, reproduced from \cite{WikiNeuron}.}
\end{figure}

The behaviour of a neuron is determined by the chemical
processes of the soma, which are characterised by the flow of many
types of ions (ion channels) through the soma membrane. This membrane
contains ion gates that can be activated and deactivated at certain
input potentials and membrane potentials to allow (or prohibit) ions
in and out of the soma. Neurons communicate with each other by
transmitting neural spikes. These spikes take the form of brief and
sudden voltage changes that are transmitted down a cell's axon to
synaptic connections with another neuron.

Although there are some arguments, it has been widely accepted that
the information transmitted by the neurons is only related with the firing
rate of the spike train, not the actual potentials, nor the shapes of
the spikes \cite{Gerstner2002}. A spike cannot be generated
immediately after the previous one. Instead, there is a refractory
period between them. Neurophysiologists divide this interval into two
periods \cite{Gerstner2002}:

\begin{itemize}
\item Absolute refractory period: the second pike cannot be generated
  regardless of any stimulation.
\item Relative refractory period: the second spike can only be
  generated if the stimulus is strong enough.
\end{itemize}

For humans, the absolute refractory period is about 1ms
\cite{Gerstner2002}, therefore, the maximum firing rate is limited to
about 1000 per second.
\subsubsection{Auditory Pathway}
\label{sec-2-4-2}
\label{sec:ch2-auditory-pathway}

A diagram of the afferent auditory pathway is illustrated in
Figure \ref{fig:afferent}. It can be seen that the afferent pathway
goes through a number of nuclei, each of which is a region that contains
certain types of cells, and finally reach the auditory cortex.

\begin{figure}[htb]
\centering
\includegraphics[width=0.7\textwidth]{./Chapter2Figs/afferent.jpg}
\caption{\label{fig:afferent}Simplified organisation of the auditory brainstem \cite{Yost1977}.}
\end{figure}
\subsubsection{Auditory Cortex}
\label{sec-2-4-3}
\label{sec:ch2-auditory-cortex}

The auditory cortex is responsible for the most difficult tasks of the
auditory system. Most importantly, for humans, it performs speech
processing allowing us to communicate with each other. In the case of
other animals; it allows them to recognise, for example, the sound of
their enemies.
\section{Auditory Models}
\label{sec-3}
\label{sec:ch2-auditory-models}

The auditory models explain how the psychology of hearing could be
understood in terms of signal processing. This section reviews a
number of computational models for the various components that make up
the auditory peripheral system.
\subsection{Head-Related Transfer Function}
\label{sec-3-1}
\label{sec:ch2-head-related-transfer-function}

Existing models of the auditory periphery usually starts from the
response of the middle ear. However, the outer ear and other parts of the
body also play an important role in the signal processing chain,
especially for the sound localisation task. The mathematical
description of the acoustic effects (reflection, diffraction, and
interference) by the outer ear and other parts of the body (head,
shoulder, torso and etc.) is referred as a transfer function. In the
frequency domain, it is called head-related transfer function (HRTF).
The HRTF $H(f)$ can be described as \footnote{Ext: ``Add reference.'' }:

\nomenclature{HRTF}{Head-related transfer function}

\begin{equation}
  \label{eq:HRTF}
  H(f) = \frac{\textrm{Out}(f)}{\textrm{In}(f)}
\end{equation}

where $f$ is the frequency input. The HRTF is the Fourier transform of
the head-related impulse response (HRIR) in the time domain.

\nomenclature{HRIR}{Head-related impulse response}

The HRTF has been widely used in sound localisation and other
applications, such as stereo-sound synthesis. However, it depends
strongly on the physical characteristics of the listener (the size and
shape of the pinna and etc.) and the acoustic environment (echos,
noises, and etc.). This makes the implementation of the HRTF quite a
challenge for a general purpose neuromorphic auditory periphery.
\subsection{Middle Ear Filter}
\label{sec-3-2}
\label{sec:ch2-middle-ear-filter}

It is reasonable to consider the middle ear as a linear system whose
input is the pressure wave level near the eardrum and the output is
the stape velocity or displacement \cite{Goode1994}. Therefore, the
middle ear is often modelled as a filter with an appropriate frequency
response. This model could be an IIR (e.g., \cite{Lopez-Poveda1996} and
\cite{Tan2003}), an FIR (e.g., \cite{Lopez-Poveda2001} and
\cite{Lopez-Najera2007}), or cascaded filters (e.g.,
\cite{Sumner2003b} and \cite{Holmes2004}). All these filter models
are possible be implemented in the neuromorphic auditory periphery.

\nomenclature{IIR}{Infinite impulse response}
\nomenclature{FIR}{Finite impulse response}
\subsection{BM Filterbank}
\label{sec-3-3}
\label{sec:ch2-bm-filterbank}

The BM is the component in the auditory periphery which has the most
number of models and varies widely in complexity. Early BM models only
simulate the basic frequency selectivity; whereas recent BM models
exhibit more nonlinear characteristics to be consistent with the
physiological observations.

\begin{enumerate}
\item Linear BM filters
\item Nonlinear BM filters
\end{enumerate}
\subsubsection{Linear BM Filters}
\label{sec-3-3-1}
\label{sec:ch2-linear-bm-filters}

A linear BM filter has a linear transfer function. It is
generally evaluated by the sharpness of the frequency response curve
at its characteristic frequency (CF) and the
highest achievable quality factor. Almost all the reported
neuromorphic auditory system (both analogue and digital) adopt linear
BM filters which are widely applied in various applications. Some
important developments are listed below.

\nomenclature{CF}{Characteristic frequency}

\begin{itemize}
\item Zweig and et al's approximate transmission line model
  (\cite{Zweig1976}),
\item Patterson and et al's gammatone filter (GTF) (\cite{Patterson1992}),
\item Lyon's all-pole asymmetric GTF (\cite{Lyon1997}),
\item Irino and Patterson's original gammachirp (GC) filter
  (\cite{Irino1997}).\footnote{Ext: ``Describe the concept for each models (4-5 lines) }
\end{itemize}


The gammatone filter (GTF) by Patterson \cite{Patterson1992} is the
most-widely applied linear BM filter, because it provides a reasonable
trade-off between the physiological correspondence and computational
efficiency. \footnote{Ext: ``What's GTF = explain'' }
\subsubsection{Nonlinear BM Filters}
\label{sec-3-3-2}
\label{sec:ch2-nonlinear-bm-filters}

A number of nonlinear BM filters have also been developed, which have
nonlinear transfer functions, such as:

\begin{itemize}
\item Goldstein's multiple bandpass nonlinear (MBPNL) filters
  (\cite{Goldstein1990}),
\item Zhang and et al.'s feedback model (\cite{Zhang2001}),
\item Irino and et al.'s compressive Gammachirp (cGC) filter
  (\cite{Irino2001}),
\item Lopez-Poveda, Meddis and et al.'s dual-resonance nonlinear
  (DRNL) filters proposed for humans, guinea ping, and chinchilla
  (\cite{Lopez-Poveda2001}, \cite{Sumner2003b} and
  \cite{Lopez-Najera2007}),
\item Irion and Patterson's dynamic compressive Gammachirp (dcGC)
  filter (\cite{Irino2006}),
\item Ferry and Meddis' DRNL filter with efferent suppression
  (\cite{Ferry2007}). \footnote{Ext: ``Describe each models'' }
\end{itemize}

\nomenclature{MBPNL}{Multiple bandpass nonlinear}
\nomenclature{cGC}{Compressive Gammachirp}
\nomenclature{dcGC}{Dynamic compressive Gammachirp}
\nomenclature{DRNL}{Dual-resonance nonlinear}

Among these nonlinear BM filters, the GC-family filters and the
DRNL-family filters are mostly referred \footnote{Ext: ``What do you mean by `referred'?'' }. Both of them have been
improved extensively during the past decade, and have the ability to
simulate various BM nonlinear phenomena, such as compression,
suppression and distortion.
\subsubsection{Discussion}
\label{sec-3-3-3}
\label{sec:ch2-discussion1}

Linear BM filters are favoured by engineers, because they have simple
structures which could be easily implemented as electronic circuits.
This has been demonstrated by the author's work of FPGA implementation
of the classic GTF \cite{Yang2009}. However, they have no ability to
simulate the nonlinear characteristic of the BM \footnote{The Ext did not understand it was my work, rephrase these two
  sentence. }. Therefore, current
auditory peripheral models usually adopt nonlinear BM filters other
than linear ones.

Compared to linear BM filters, nonlinear BM filters require much more
computation time and resources. This gets worse when tens or hundreds
of nonlinear filters are grouped as a filterbank to model the BM.
Generally speaking, a software implementation of a bank
nonlinear BM model cannot run in real-time on the common sequential
computer processor, let alone current embedded microprocessors.

Based on the review, there were no real-time digital implementations
of nonlinear BM filters before this work. Therefore, it could be an
important contribution to explore the possibility of implementing an
real-time neuromorphic auditory periphery which adapts nonlinear BM
filters using current digital electronic technology.
\subsection{IHC Models}
\label{sec-3-4}
\label{sec:ch2-ihc-models}

The IHCs account for the mechanoelectrical and electrochemistric
processes within the cochlea. The input to an IHC model is the
instantaneous amplitude of the BM site attached to that IHC, the
output being the neurotransmitters released. The general interaction
between the IHC and AN is illustrated in Figure \ref{fig:an-synapse}:

\begin{enumerate}
\item The stereocilia opens the icon channels that allow potassium
   (K$^{+}$) and calcium (Ca$^{\mathrm{+2}}$) from the endolymph to
   enter the cell.
\item The IHC get depolarised and generates a receptor potential (RP)
   in return.
\item This RP opens Ca$^{\mathrm{+2}}$ channels to allow
   Ca$^{\mathrm{+2}}$ enter the cell.
\item Vesicles which contains neurotransmitters from the IHC are
   released, then cross the synapse cleft between the IHC and an AN
   receptor.
\item The AN fibre generates an action potentials (AP) which results a
   neural spike.
\end{enumerate}

\nomenclature{K$^{\mathrm{+}}$}{Potassium}
\nomenclature{Ca$^{\mathrm{+2}}$}{Calcium}
\nomenclature{RP}{Receptor potential}
\nomenclature{AP}{Action potential}

\begin{figure}[htb]
\centering
\includegraphics[width=1.0\textwidth]{./Chapter2Figs/an-synapse.eps}
\caption{\label{fig:an-synapse}Illustration of the general process of IHC stimulating neural spikes in the synapse, reproduced form \cite{WikSynapse}.}
\end{figure}

The general model structure, whose mechanism is responsible for this
interaction between IHCs and AN fibres, can be seen in
Figure \ref{fig:ihc-model}. This general model comprises a number of
reservoirs that contain a transmitter substance. Existing IHC models
can be divided into three types.

\begin{figure}[htb]
\centering
\includegraphics[width=1.0\textwidth]{./Chapter2Figs/ihc-model.eps}
\caption{\label{fig:ihc-model}Generalised representation of IHC models.}
\end{figure}
\subsubsection{Single-Reservoir Models}
\label{sec-3-4-1}
\label{sec:ch2-signle-reservoir-models}

The first IHC models developed by Schroeder \cite{Schroeder1974}
consisted of a single reservoir that released the transmitter to stimulate
the pose-synaptic membrane. The amount of released transmitter
depended on the stimulus-related permeability of the cell membrane.
Such a simple model is unable to represent the two-component
adaptation known to exist in the IHC.
\subsubsection{Multiple-Reservoir Models}
\label{sec-3-4-2}
\label{sec:ch2-multiple-reservoir-models}

More sophisticated models emerged such as that of Schwid
\cite{Schwid1982}, Ross \cite{Ross1982} and its improvement
\cite{Ross1996}.

The model by Schwid \cite{Schwid1982} consisted of independent
immediate stores of transmitters. The reservoirs were ordered in
increasing threshold such that a transmitter can be released from only 1
reservoir at low stimulus levels and by all 6 at high stimulus levels.
These reservoirs were arranged like the immediate stores in Figure
\ref{fig:ihc-model}.

The model by Ross \cite{Ross1982} employed 4 reservoirs in series.
The permeability between the reservoirs is constant and so the flow rate
between them is determined by the concentration or level differences.
The transmitter release from the final reservoir is governed by 2
permeabilities. The first permeability is fixed and the second is
related to the stimulus intensity.
\subsubsection{Feedback Models}
\label{sec-3-4-3}
\label{sec:ch2-feedback-models}

The model proposed by Meddis (\cite{Meddis1986}, \cite{Meddis1988})
consisted of 3 reservoirs that incorporated a novel re-uptake and
re-synthesis loop. The Meddis model has been the most computationally
efficient model until today, which incorporates the three main
properties (see \ref{sec:ch2-audisys-haircell}) of an IHC.
However, the original Meddis IHC model only represents the average
firing rate of HSR fibres, since the majority of auditory nerve fibres
are this type. Current developments in physiology show different
adaptations in the three types of fibre (HSR, MSR, and LSR). In this
case, Sumner has reported a revised version of the Meddis IHC model
(\cite{Sumner2002} and \cite{Sumner2003a}) to be consistent with the
physiological findings.
\subsubsection{Discussion}
\label{sec-3-4-4}
\label{sec:ch2-discussion2}

The original Meddis IHC model has been the most important development
in the research of IHC modelling. Although Sumner's revised Meddis
model is the most philological corresponding IHC model at present, it
is not appropriate to be implemented as a module in the neuromorphic
system because of its sophisticated arithmetics. The original Meddis
model has been successfully implemented both in analogue and digital
electronic circuits. These related works will be discussed in the next
chapter.
\subsection{AN Model}
\label{sec-3-5}
\label{sec:ch2-an-model}

In physiology, the actual process of the transformation from the
release of neurotransmitters to AN spikes has not been known. However,
the release rate has been used as an indication of the firing rate of the
AN fibres attached to the synapse between an IHC and a bunch of AN
fibres. There are two forms of representation of the IHC model output.
One is the spike probability, which is the same for all the AN fibre
in that synapse, the other is presented by several spike trains for
all AN fibres in the synapse. The former is a general description and
runs quicker, while the later needs more computation time but is
essential for other neural models in the brainstem. It should be noted
that the refractory period (see \ref{sec:ch2-audisys-neuron})
need to be considered for both representation of the AN output. This
means that the interval time between two spikes should be greater than
about 1ms, and the maximum firing rate is about 1kHz. \footnote{Ext: ``Rephrase this paragraph'' }

It is obvious that this research need the second type output from the
AN model. The generation of neural spikes is a stochastic process and
therefore requires random (or pseudo-random) numbers to generate
uncorrelated spiking patterns in the auditory nerve. Pseudo-random
numbers can be efficiently generated using linear-feedback
shift-registers (LFSRs) \cite{George2007}.

Generally, there are two approaches to generate auditory spikes. One
uses binomial distribution method, while the other uses the rejection
method \footnote{Ext: ``Reference required'' Need to explain the rejection method
  before this conclusion. }.
\subsubsection{Binomial Distribution Method}
\label{sec-3-5-1}
\label{sec:ch2-binomial-distribution-method}

This method uses binomial statistics together with a random number
generator. The binomial distribution method has been applied in the
revised Meddis IHC model to generate spikes for each AN fibre.
However, this method usually employs logarithmic, exponential or
division functions each of which are expensive to implement in digital
technology.
\subsubsection{The Rejection Method}
\label{sec-3-5-2}
\label{sec:ch2-the-rejection-method}

In this case, assume $n$ is a random number whose value is between 0 and 1
inclusive, $p$ is the probability value produced by the IHC. If $n\ge
p$, the corresponding nerve fibres fires. For building a digital
neuromorphic auditory periphery, it is also a very economic method.
Therefore, the rejection method is chosen for this research.
\section{Summary}
\label{sec-4}
\label{sec:ch2-summary}

In terms of signal processing, the auditory periphery is a complicated
system which consists many stages of processing according to its
anatomical structure. An appropriate model has to be developed for
building a digital neuromorphic system which would have the ability to
represent some useful functions of the biological auditory system.
In this chapter, some widely-accepted auditory models are evaluated
according to their physiological correspondence and computational
efficiency in order to form an initial structure of the system, as
illustrated in Figure \ref{fig:init-design}. The selected models for
different components of the neuromorphic system are listed in
Table \ref{tbl:init-models} \footnote{Ext: ``you need to discuss this table, with more words'' }. The investigation and implementation of
these components will be described in later chapters \footnote{Ext: ``Refer to specific chapters, which chapters.'' }, and the finial
system structure is not the same as this one.

\begin{figure}[htb]
\centering
\includegraphics[width=0.6\textwidth]{./Chapter2Figs/init-design.eps}
\caption{\label{fig:init-design}Initial design of the system structure.}
\end{figure}

\begin{longtable}{|c|l|l|}
\\
 Components           &  Description           &  Model                \\
\hline
\endhead
\hline\multicolumn{3}{r}{Continued on next page}\
\endfoot
\endlastfoot
\hline
 Ourter ear and etc.  &  HRTF                  &  A dummy listener      \\
 Middle ear           &  An IIR filter         &  No predefined model   \\
 BM                   &  An bank of BM filters  &  Linear or nonlinear filters  \\
 IHCs                 &  A group of IHC models  &  The Meddis IHC model  \\
 AN fibres            &  A group of AN fibre models  &  Rejection method      \\
\caption{Selected auditory models for different components of the neuromorphic system.} \label{tbl:init-models}\end{longtable}


\setstretch{1}

\bibliographystyle{plain}
\bibliography{bib/thesis}

\end{document}
